\begin{algorithm}[ht]
\caption{Transformation procedure}
\label{alg:transformation}
\begin{algorithmic}[1]

\REQUIRE{InSynth tree, appropriate $\lambda$ declarations}

\STATE{let $r$ be the root of the InSynth tree} \COMMENT{the root, query node,
has type $(? \rightarrow \bot)$}

\RETURN{Transform($\emptyset$, $r$)}

\end{algorithmic}
\end{algorithm}

The transformation starts with the root node $r$ and an empty typing context
(environment).
The algorithm recursively calls the $Transform$ procedure given in Algorithm
\ref{alg:transform_call}.

% \floatname{algorithm}{Procedure}
% \renewcommand{\algorithmicrequire}{\textbf{Input:}}
% \renewcommand{\algorithmicensure}{\textbf{Output:}}

\begin{algorithm}[ht]
\caption{Transform}
\label{alg:transform_call}
\begin{algorithmic}[1]

\REQUIRE{
	%term $t$, 
	context $\Gamma$, node $n$}
% \COMMENT{The procedure requires:}
% \STATE\COMMENT{since the call is recursive, the term resolved in this
% procedure $x$, will be applied to $t$, i.e. we will have $\lambda$ expression $(t x)$}
\STATE\COMMENT{context $\Gamma$ is the current typing context when resolving the
current term $x$} \STATE\COMMENT{node $n$ is the current node (InSynth SimpleNode) which provides information about resolvent of the current goal type}

\IF{goal is of type $(X \Rightarrow Y)$}
\FORALL{nodes $n'$ with goals of type $Y$}
\FORALL{parameter types $X_i$ in $X$ according to real type}
\STATE{let $x_i$ be fresh variable of type $X_i$}
\ENDFOR
\STATE{let $a$ be an abstraction which takes all parameters in
$X$}
\COMMENT{$a = (\lambda x_1:X_1. (\lambda x_2:X_2. \ldots (\lambda
x_n:X_n.\textrm{ ``\_``})))$}
\FORALL {$t'$ in Transform(\Gamma $\cup (\bigcup_i x_i:X_i)$, $n'$)}
\RETURN {plug $t'$ into the abstraction $a$}
\COMMENT {in place of ``\_``}
\ENDFOR
\ENDFOR
\ENDIF
\IF{goal is of type $X$}
\FORALL{declaration $d$ in declarations in $n$}

\IF{type of $d$ is $X$}
\STATE{return expression of $d$}
\ELSIF{type of $d$ is $X_1 \Rightarrow X_2 \Rightarrow \ldots \Rightarrow X_n
\Rightarrow X$}
\FOR{$i$ in $1..n$}
\IF{there is a leaf child $n'$ of node $n$}
\STATE{let $S_i$ be the set of all variables $v$ such that $v:X_i \in \Gamma$}
\ENDIF
\FORALL{child $n'$ nodes of $n$ with goal type of $X_i$}
% \STATE{find child $n'$ of node $n$ with goal type of $X_1$}
\STATE{let $S_i =$ $S_i$ $\cup$ Transform($\Gamma$, $n'$)}
\ENDFOR
\ENDFOR
\FORALL{combination $x_1, x_2, \ldots, x_n$ of elements $x_1 \in S_1, x_2 \in
S_2, \ldots, x_n \in S_n$}
\RETURN{$x_1$ $x_2$ $\ldots$ $x_n$}
\COMMENT{application of $x_n$ to $x_1$ $\ldots$ $x_{n-1}$}
\ENDFOR
\ENDIF
\ENDFOR
\ENDIF

\end{algorithmic}
\end{algorithm}

\textit{Note:} The line $return$ $x$ in the procedure $Transform$ does not
designate a conventional return call from a function, but rather an indication that such
result $x$ should be returned as one of the results (more
specifically, $Transform$ returns a set of terms which contains all terms which
are prepended with the $return$ command in the Algorithm \ref{alg:transform_call}).